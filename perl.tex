\chapter{Perl}
Perl adalah bahasa pemrograman kesukaan saya.
Pada mulanya, sekitar akhir tahun 80-an, Perl merupakan 
satu-satunya bahasa pemrograman yang {\em portable}.
Saya menggunakan banyak komputer dengan sistem operasi yang
berbeda-beda; MS-DOS, UNIX dan variasinya (SunOS, Solaris, AIX,
SCO UNIX, HP-UX, Linux, FreeBSD, OpenBSD).
Sesungguhnya ada bahasa lain yang juga tersedia di berbagai
platform , yaitu bahasa {\em C}, tetapi bahasa C harus dirakit
({\em compile}) terlebih dahulu, sementara bahasa {\em Perl} 
adalah intrepreter yang langsung dapat dieksekusi tanpa melalui proses rakit.
Siklus pengembangan program menjadi lebih cepat.

Cara saya memprogram dalam bahasa Perl, sesungguhnya
seperti cara saya membuat program program dalam bahasa {\em C}.
Saya lebih menekankan aspek {\em readability} daripada
membuat program yang singkat dan padat.
Padahal banyak programmer Perl yang cenderung membuat programnya
singkat, bahkan sebaris yang disebut {\em one-liner}.

Saya sering menggunakan bahasa Perl. Banyak hal yang sudah teringat 
di kepala saya sehingga tidak membutuhkan catatan ini.
Oleh sebab itu mungkin saja catatan ini terlalu ``meloncat''.

Salah satu yang paling dibenci oleh orang (bukan programmer Perl)
terhadap bahasa Perl adalah penggunaan dolar (\$) atau karakter
lainnya (@, !, dan seterusnya) untuk variabel yang membuat kode terlihat
seperti banyak cacing. ha ha ha.

\section{Variabel}
Variabel di perl ada beberapa jenis dan ini ditunjukkan dengan
karakter di depan nama variabelnya.
\begin{itemize}
\item Dolar \texttt{\$} untuk skalar. Contoh \texttt{\$x = 'A'}, 
   \texttt{\$x[0]=123},
   dan seterusnya. Tipe dari variabel tersebut bebas,
   bisa numerik (integer) atau string.
   (Ini enaknya perl dan juga sekaligus bahayanya.)
\item Tanda {\em at} \texttt{@} untuk array. Contoh \texttt{@x}.
   Elemen dari array tersebut dapat diakses dengan 
   tamda dolar \texttt{\$x[i]}.
\item Persen (\%) untuk {\em associative array}. Contoh \texttt{\%x}.
   Elemen diakses dengan menggunakan \texttt{\$x\{key\}}.
   Ini adalah salah satu hal yang paling asyik dari perl.
   Saya bisa menggunakan apa saja (umumnya string) sebagai {\em key}
   dari {\em associative array} ini. Contoh 
   \texttt{\$header\{Nama\}="Budi Rahardjo"},
   \texttt{\$header\{Twitter\}="@rahard"}
\end{itemize}

\section{Loop}
Salah satu hal yang sering dilakukan oleh sebuah program adalah
membuat pengulangan (loop).
Jika kita mengetahui jumlah loop, maka biasanya {\em for-loop}
merupakan construct yang paling lazim.
Bagi yang familier dengan bahasa {\em C}, bentuk ini mirip sekali
dengan loop di bahasa {\em C}.
\begin{lstlisting}
for ($i=0; $i<10; $i++) {
   print "$i ... ";
}
\end{lstlisting}

Look juga digunakan untuk memproses sebuah berkas, membaca baris
per-baris dari baris pertama hingga akhir berkas (\texttt{EOF}).
Jika kita memiliki skrip \texttt{baca.pl} dengan isi seperti
di bawah ini.
\begin{lstlisting}
while (<>) {
   print $_;
}
\end{lstlisting}
Kemudian dijalakan dengan memberikan nama berkas yang akan diproses,
\texttt{berkas-data.txt}. Berkas tersebut akan dibaca baris per-baris
dan akan ditampilkan (print).
Variabel \texttt{\$\_} berisi baris yang dibaca.
\begin{lstlisting}
perl baca.pl berkas-data.txt
\end{lstlisting}

\section{Topic Generator}
Berikut ini adalah sebuah contoh skrip yang saya beri nama
\texttt{generator.pl}.
Skrip ini terinspirasi dari kesulitan orang dalam mencari topik
untuk menulis di blognya. Skrip ini menampilkan (generate) topik.
Sesungguhnya skrip ini hanya membaca topik-topik yang sudah dituliskan
dalam berkas \texttt{topics.txt}, kemudian memilih secara random
salah satu dari baris (topik) tersebut.
Setiap topik dipisahkan dengan dua garis (dash).

Contoh (cuplikan) isi berkas \texttt{topics.txt} adalah 
sebagai berikut. Anda tinggal menambahkan topik baru ke
dalam berkas tersebut. Skrip akan langsung menggunakannya
sebagai bagian dari pilihan.
\begin{lstlisting}
Ceritakan tentang buku yang paling berkesan kepada Anda
--
Ceritakan tentang media sosial yang paling sering Anda gunakan.
Mengapa Anda sering menggunakan itu?
--
Jika Anda menjadi superhero, siapa yang Anda pilih?
--
\end{lstlisting}

Berikut ini adalah kode \texttt{generate.pl} tersebut.
Penjelasan akan saya berikan di bawah.

\lstset{
  numbers=left,
  firstnumber=1,
  numberfirstline=true
}
\lstinputlisting[language=Perl]{generate2.pl}
\lstset{numbers=none}

Baris pertama hanya menunjukkan dimana interpreter perl berada.
Baris 2, set nama berkas topik ke variabel \texttt{\$topicDB}.
Baris 3, membuka berkas tersebut dalam mode read
(lihat tanda $<$ ) dan apabila gagal akan menampilkan
pesan (\texttt{die \$!}).
Ini adalah salah satu hal yang menarik bagi saya,
yaitu error message akan ada pada variabel itu.
Tinggal di-print saja.
(Tentu saja ini bagus untuk command line script,
tapi tidak bagus kalau dijadikan aplikasi web-based.)

Baris ke 4, menyiapkan diri untuk memulai menghitung
jumlah topik (ke dalam variabel \texttt{\$count}).
Baris ke 5, melakukan looping terhadap berkas topik itu.
Jika menemukan baris yang dimulai dengan garis dua
(--) maka counter perlu dinaikkan.
Baris ke 6, memasukkan kalimat (baris) ke dalam 
associative array \texttt{\%topic} (dengan nomor topik
sebagai key).
Dia dibuat begitu agar kalau ada topik yang lebih dari
satu baris, digabung menjadi satu variabel.

Baris ke 9, memilih salah satu dari topik secara random.
Angka random dipilih maksimal sama dengan jumlah
topik.
Baris ke 10, menampilkan di layar. Mudah bukan?

\section{To Do}
Hal-hal yang belum saya catat adalah tentang {\em search and replace}
dan {\em regular expression} (regex).

