\chapter{Git}
Git\footnote{Client git dapat diperoleh dari https://git-scm.com/\\
Sementara server git dapat menggunakan Gogs yang dapat diperoleh di
https://gogs.io/}
adalah sebuah sistem untuk melakukan {\em versioning} dari dokumen,
yang biasanya adalah kode sumber ({\em source code}).
Biasanya {\em git} digunakan untuk mengembangkan software secara bersama-sama.

Buat direktori tempat bekerja. Pindah ke direktori tersebut dan
melakukan inisialisasi {\em git}. 
Pada direktori tersebut akan dibuat direktori \texttt{.git}.

\begin{lstlisting}
git init
ls -a
\end{lstlisting}

Untuk mengetahui konfigurasi git Anda secara global dapat digunakan
perintah {\em list} berikut. Untuk mengubah konfigurasinya juga dapat
dilakukan dengan perintah \texttt{git config}.
\begin{lstlisting}
git config --list
git config --global user.name "budi rahardjo"
git config user.name
\end{lstlisting}

Selanjutnya kita mulai menambahkan server git (dalam contoh ini
IP-nya adalah 192.168.1.1 - untuk kasus Anda coba cari alamat ini,
misal di github.com ada di sebelah kanan atas).
Kemudian kita dapat menarik kode dengan perintah {\em pull}.

\begin{lstlisting}
git remote add origin http://192.168.1.1/somepath/whatever.git
git pull origin master
\end{lstlisting}
