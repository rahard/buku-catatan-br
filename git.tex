\chapter{Git}
Git\footnote{Client git dapat diperoleh dari https://git-scm.com/\\
Sementara server git dapat menggunakan Gogs yang dapat diperoleh di
https://gogs.io/}
adalah sebuah sistem untuk melakukan {\em versioning} dari dokumen,
yang biasanya adalah kode sumber ({\em source code}).
Biasanya {\em git} digunakan untuk mengembangkan software secara bersama-sama,
tetapi dia dapat digunakan untuk keperluan lain.
Penulisan buku ini menggunakan {\em git} untuk sinkronisasi
antar komputer tempat saya bekerja.

Selain untuk mengembangkan kode bersama, git juga bermanfaat untuk
mengembangkan kode di komputer yang berbeda.
Di komputer A, saya bisa bekerja. Setelah selesai, dokumen saya
simpan ({\em push}) ke server git. 
Di komputer B, saya bisa mengambil berkas atau perubahan terbaru
dengan melakukan {\em pull}. Maka pekerjaan di kedua komputer tersebut
menjadi tersinkronisasi dengan {\em git} sebagai penengah.

Mari kita mulai menggunakan {\em git}.
Buat sebuah direktori tempat bekerja. Pindah ke direktori tersebut dan
melakukan inisialisasi {\em git}. 
Pada direktori tersebut akan dibuat direktori \texttt{.git}
yang berisi informasi mengenai pekerjaan kita. 
Untuk saat ini, kita tidak perlu tahu detail dari isi direktori itu.

\begin{lstlisting}
git init
ls -a
\end{lstlisting}

Untuk mengetahui konfigurasi git Anda secara global dapat digunakan
perintah {\em list} berikut. Untuk mengubah konfigurasinya juga dapat
dilakukan dengan perintah \texttt{git config}.
Konfigurasi ini hanya perlu dilakukan sekali saja.
\begin{lstlisting}
git config --list
git config --global user.name "budi rahardjo"
git config user.name
\end{lstlisting}

Selanjutnya kita mulai menambahkan server git (dalam contoh ini
IP-nya adalah \texttt{192.168.1.1} - untuk kasus Anda coba cari alamat ini,
misal di github.com ada di sebelah kanan atas).
Kemudian kita dapat menarik kode dengan perintah {\em pull}.

\begin{lstlisting}
git remote add origin http://192.168.1.1/nama-proyek/proyeknya.git
git pull origin master
\end{lstlisting}

Boleh jadi kita memulai dengan melakukan {\em clone} juga.
Contoh di bawah ini kita akan membuat direktori \texttt{libgit2}
dan mengambil semua berkas yang ada di sana.
\begin{lstlisting}
git clone https://github.com/libgit2/libgit2
\end{lstlisting}

Sekarang kita boleh bekerja dengan berkas-berkas.
Setelah selesai, berkas perubahaan kita bisa ditambahkan
dengan memberikan perintah {\em add}.

\begin{lstlisting}
git add somefile
git add *.c
git add LICENSE
git commit -m 'initial project version'
\end{lstlisting}
Pilihan \texttt{-m} tersebut akan mendokumentasikan pesan ke dalam
perubahan. Jika kita tidak menggunakan opsi tersebut, kita akan
dibawa ke editor pilihan kita.

Setor perubahan ke server git.

\begin{lstlisting}
git push origin master
\end{lstlisting}

Untuk memulai lagi, misal hari berikutnya untuk bekerja.
Contoh berikut menunjukkan bahwa apa yang ada di direktori kita
sama dengan ada yang di (remote) server.
\begin{lstlisting}
git pull origin master
From https://github.com/rahard/buku-catatan-br
 * branch            master     -> FETCH_HEAD
Already up-to-date.
\end{lstlisting}

Sekali-sekali saya cek apakah saya membuat perubahan yang perlu
saya setor. Ada daftar berkas yang berubah dan perlu saya
tambahkan dengan perintah add sebelum di-commit.
\begin{lstlisting}
git status
git add fileyangberubah
git commit -m 'perubahan ini tentang apa'
\end{lstlisting}
