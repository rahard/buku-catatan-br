\chapter{Postgres}
Postgres (postgresql) merupakan program database yang banyak digunakan
untuk aplikasi internet. Ini merupakan saingan dari MySQL.

Untuk mengakses database Postgres dari shell dapat digunakan
program {\em psql}. Ini merupakan bagian dari klien Postgress\footnote{
(Untuk klien di Mac OS X, saya masih belum menemukan yang pas
karena saya menggunakan {\em Homebrew} sebagai
package manager. Ternyata klien Postgres adanya di Fink.
Sementara ini saya menggunakan {\em PSequel}, yang GUI,
untuk Mac OS X.)
}.

%\begin{lstlisting}[frame=single]
\begin{lstlisting}
psql -U username -h 192.168.1.1 -d databasename 
\end{lstlisting}

Untuk menjalankan kode (perintah SQL) yang ada dalam berkas
``perintah.sql'' dapat dilakukan dengan cara di bawah ini.
Keluaran akan ditampilkan di {\em stdout}, yang kemudian bisa
di-pipe ke program lain (atau diarahkan ke file).

\begin{lstlisting}
psql -U username -h 192.168.1.1 -d databasename -f perintah.sql
\end{lstlisting}

Isi berkas ``perintah.sql'' adalah seperti ini:

\begin{lstlisting}
select * from auth_user
\end{lstlisting}
