\chapter*{Pengantar}

Seringkali saya mengerjakan pemrograman dan lupa akan sesuatu.
Dulu saya pernah melakukan ini, tapi kok sekarang lupa lagi.
Lantas saya mencari-cari di komputer saya, kalau-kalau ada contoh
kode atau catatan yang pernah saya buat.
Atau kemudian saya cari-cari di internet untuk contoh yang saya maksudkan.

Proses ini menghabiskan waktu.
Daripada seperti itu, lebih baik saya catat di sini saja.
Selain bermanfaat untuk saya, mudah-mudahan ini juga dapat bermanfaat
bagi orang lain.

Tentu saja format dan isi yang saya tuliskan ini sesuai dengan kebutuhan
saya. Misalnya, contoh perintah-perintah yang ditampilkan adalah apa-apa
yang pernah atau biasa saya gunakan.
Ada banyak perintah (command) dan variasinya.
Bagi saya, hal-hal yang khas (spesifik) saya ini yang justru perlu saya
catat. Jadi ketika butuh, saya teringat pernah melakukan itu tetapi
lupa {\em command line}-nya. Maka di sinilah fungsi dari buku atau
catatan ini. Untuk yang belum pernah saya lakukan, ya saya akan cari
dari internet atau membaca manualnya.

Topik yang masuk ke kategori ``programming'' ini cukup luas.
Jadi ada banyak hal yang mungkin agak sedikit melebar, meskipun
bisa juga dikait-kaitkan.
Sebagai contoh ada pembahasan tentang \texttt{git} 
({\em source code versioning}) dan database (\texttt{Postgres}).

Saya suka menggunakan {\em command line}. 
(Bahkan untuk kodingpun saya masih suka menggunakan editor \texttt{vi})
Untuk itu contoh-contoh yang saya tampilkan adalah yang menggunakan
{\em command line}.
Alternatif cara yang menggunakan GUI (graphical user interface)
sebetulnya ada dan banyak. Mungkin nanti saya berikan tautan
ke halaman atau informasi yang berhubungan dengan itu.

Platform saya juga bervariasi. Umumnya saya menggunakan
Linux mint (untuk desktop), Mac OS X (untuk portable / jalan), dan
Linux Debian (dalam virtualbox di Mac OS)
Ada sedikit kerepotan tentang tools yang tersedia di platform
yang berbeda-beda.

Selamat menikmati\\
Bandung, 2016\\
Budi Rahardjo\\
\texttt{@rahard}

